\documentclass[final,xcolor=table,svgnames]{beamer}
\usepackage[english]{babel}
%\usepackage[latin1]{inputenc}
\usepackage[T1]{fontenc}
\usepackage{graphicx}
\usepackage{hyperref}
\usepackage{url}
\usepackage{tikz}
\usepackage{doi}
\usepackage{booktabs}
\usepackage{fontawesome}
\usepackage{academicons}
\usepackage{fontspec}
\usepackage{colortbl}


\setmainfont{Gillius ADF No2}

\newcommand{\logoheight}{1cm}

\DeclareGraphicsExtensions{.pdf,.png,.PNG,.JPG,.jpg,.jpeg,.gif}
\graphicspath{
{./figures/},
{./figures/scientificprofiles/}
{/home/ctroupin/Presentations/figures4presentations/logo/},
}


\setbeamertemplate{navigation symbols}{}
\setbeamertemplate{items}[square] 
\setbeamertemplate{caption}[numbered]
%\setbeamerfont{caption}{size=\scriptsize,family=\it}

%\usefonttheme{professionalfonts} % using non standard fonts for beamer
%\usefonttheme{serif} % default family is serif

\usetikzlibrary{arrows,shapes,backgrounds}
\tikzstyle{every picture}+=[remember picture]
\tikzstyle{na} = [baseline=-.5ex]

\definecolor{bluegher}{HTML}{4E519F}
\definecolor{mygrey}{rgb}{0.75,.75,.75}
\definecolor{arrowcolor}{rgb}{0.1,0.1,0.1}
\definecolor{alertbg}{HTML}{FEFFBA}

\setbeamercolor{title}{fg=white}
\setbeamercolor{institute}{fg=mygrey}
\setbeamercolor{frametitle}{fg=white,bg=bluegher}
\setbeamercolor{structure}{fg=bluegher}
\setbeamercolor{item projected}{fg=black}
\setbeamercolor{alerted text}{fg=bluegher,bg=alertbg}

\setbeamercovered{invisible}
\setbeamertemplate{items}[triangle] 

\newenvironment{boxalertenv}{\begin{altenv}%
      {\usebeamertemplate{alerted text begin}\usebeamercolor[fg]{alerted text}\usebeamerfont{alerted text}\colorbox{bg}}
      {\usebeamertemplate{alerted text end}}{\color{.}}{}}{\end{altenv}}

\newcommand<>{\boxalert}[1]{{%
  \begin{boxalertenv}#2{#1}\end{boxalertenv}%
}}


\setbeamerfont{sectiontitle1}{size=\huge,family=\rmfamily}
\setbeamerfont{sectiontitle2}{size=\fontsize{40}{20}\selectfont,shape=\itshape,family=\rmfamily}
\setbeamerfont{author}{size=\large,family=\rmfamily}
\setbeamerfont{institute}{size=\normalsize,family=\rmfamily}
\setbeamerfont{title}{size=\fontsize{25}{20}\selectfont,shape=\itshape,family=\rmfamily}

\setbeamertemplate{enumerate items}[square]
\setbeamercolor{item projected}{bg=alertbg,fg=bluegher}

\setbeamertemplate{title page}{%
\begin{tikzpicture}[remember picture,overlay]
\fill[bluegher]
  (current page.west) rectangle (current page.south east);
\node[anchor=east] 
  at ([yshift=-30pt, xshift=-10pt]current page.north east) (author)
  {\parbox[t]{.6\paperwidth}{\raggedleft%
   \usebeamerfont{author}{\insertauthor}}};
\node[anchor=north east] 
  at ([yshift=-70pt, xshift=-10pt]current page.north east) (institute)
  {\parbox[t]{.78\paperwidth}{\raggedleft%
    \usebeamerfont{institute}\usebeamercolor[fg]{institute}{\insertinstitute}}};
\node[anchor=south west] 
  at ([yshift=20pt, xshift=20pt]current page.west) (logo)
  {\parbox[t]{.19\paperwidth}{\centering%
    \usebeamercolor[fg]{titlegraphic}\inserttitlegraphic}};
\node[anchor=east]
  at ([yshift=-55pt,xshift=-20pt]current page.east) (title)
  {\parbox[t]{\textwidth}{\raggedleft%
   \usebeamerfont{title}\usebeamercolor[fg]{title}\inserttitle}};
\end{tikzpicture}
}

%\setbeamersize{text margin left=1cm}

\newlength{\logoH}
\setlength{\logoH}{1cm}
\newcommand{\montant}{\rule{0pt}{3ex}}

%--------------------------------
\hypersetup{bookmarksopen=true,
bookmarksnumbered=true,  
pdffitwindow=true, 
pdfstartview=Fit,
%pdfpagemode=FullScreen,
pdffitwindow=true,
pdftoolbar=true,
pdfmenubar=true,
pdfwindowui=true,
pdfsubject={ODIP, Data analysis, Software citation},
pdfauthor={C. Troupin, C. Muñoz, S. Watelet, A. Barth, J.M. Beckers},
bookmarksopenlevel=0,
colorlinks=true,
linkcolor=bluegher,anchorcolor=black,%
citecolor=bluegher,filecolor=black,%
menucolor=black,urlcolor=bluegher,%
pdfpageduration=1,%
}


\title{Software citation \& \newline process traceability}
%\subtitle{}
\author[C.~Troupin]{C.~Troupin$^{\star}$, C.~Muñoz$^{\ast}$\\ Contributions by M.A.~Rújula$^{\ast}$, J.~Fernández$^{\ast}$, A.~Barth$^{\star}$ \& J.-M.~Beckers$^{\star}$}

\institute{$^{\star}$GHER - University of Liège

$^{\ast}$SOCIB Data Centre}

\date{Galway (Ireland), 2-6 October, 2017}
\titlegraphic{\includegraphics[height=\logoheight]{logo_odip2}
\includegraphics[height=\logoheight]{logo_uliege.jpeg}\\
\includegraphics[height=\logoheight]{logo_gher}
\includegraphics[height=\logoheight]{logo_socib}}
 
 \tikzset{
    vertex/.style = {
        circle,
        fill = black,
        outer sep = 0pt,
        inner sep = 1pt,
    }
}

\begin{document}

\begin{frame}[c]
\maketitle
\end{frame}

%-----------------------------------------------------------------------------------------------
\section{Context of software citation}
%-----------------------------------------------------------------------------------------------

\begin{frame}
\begin{tikzpicture}[remember picture,overlay]
\fill[bluegher]
  (current page.west) rectangle (current page.south east);
\node[anchor=east] 
  at ([yshift=35pt, xshift=-20pt]current page.east) (title1)
  {\parbox[t]{.6\paperwidth}{\raggedleft%
   \usebeamerfont{sectiontitle2}{Motivations:}}};
\node[anchor=east]
  at ([yshift=-35pt,xshift=-20pt]current page.east) (title2)
  {\parbox[t]{\textwidth}{\raggedleft%
   \usebeamerfont{sectiontitle1}\usebeamercolor[fg]{title}{Reproducibility \& Traceability}}};
\end{tikzpicture}

\end{frame}

%--------------------------------------------------------------------------------
\begin{frame}
\begin{tikzpicture}

%\node[anchor=south west,inner sep=0] (image) at (0,0) {\includegraphics[width=0.9\textwidth]{figures/im_back.png}};
     
%\draw[help lines,xstep=1,ystep=1] (0,0) grid (10,10);
%\foreach \x in {0,1,...,9} { \node [anchor=north] at (\x,0) {\x}; }
%\foreach \y in {0,1,...,9} { \node [anchor=east] at (0,\y) {\y}; }

\onslide<1>{
\node[fill=LightYellow,text=black] (OceanObservation)  at (1,7) {Ocean Observation};
\node[fill=LightGreen,text=black] (Science) at (9,7) {Science};
\node[fill=LightYellow,text=black] (GenerateDataset) at (1,5) {Generate Dataset};
\draw[->,>=stealth] (OceanObservation) edge (GenerateDataset);

\node[fill=LightSkyBlue,text=black] (Software) at (5.5,5) {Software tool};
\node[fill=LightGreen,text=black] (Publication) at (9,5) {Publication};
\draw[->,>=stealth] (Science) edge (Publication);
\draw[->,>=stealth] (GenerateDataset) edge node[anchor=center, above] {\scriptsize Analysis} (Software);
\draw[->,>=stealth] (Software) edge node[anchor=center, above] {\scriptsize Results} (Publication);

\node[fill=LightYellow,text=black] (DOIdata) at (1,3.5) {\aiDoi};
\node[fill=LightGreen,text=black] (DOIpub) at (9,3.5) {\aiDoi};
\draw[->,>=stealth](GenerateDataset) edge (DOIdata);
\draw[->,>=stealth](Publication) edge (DOIpub);
\node[fill=LightSkyBlue,text=black] (SourceCode) at (8,2) {Source Code};
\draw[->,>=stealth](Software) edge[in=90,out=0] (SourceCode);
}

\onslide<1-2>{
\node[fill=DarkOrange,text=black] (GitHub) at (4.5,.5) {\faGithub};
\node[fill=LightSkyBlue,text=black] (SourceCode) at (8,2) {Source Code};
\draw[->,>=stealth](SourceCode) edge[in=0,out=-90] node[anchor=center, above, sloped] {\scriptsize Upload} (GitHub);
\node[fill=DarkOrange,text=black] (ORCID) at (1,.5) {\aiOrcid};
\node[fill=LightSkyBlue,text=black] (Zenodo) at (4.5,2) {\includegraphics[height=.4cm]{figures/zenodo-black-200.png}}; 
\draw node[vertex] (Joint) at (2.5,0.5) {};
\draw[->,>=stealth](Software)(Joint) edge node[anchor=center, above,sloped] {\scriptsize Login} (Zenodo);
\draw[-] (ORCID) to (Joint);
\draw[-] (GitHub) to (Joint);
\node[fill=LightSkyBlue,text=black] (DOIcode) at (4.5,3.5) {\aiDoi};
\path[->,>=stealth]
(Zenodo) edge (DOIcode)
(SourceCode) edge node[anchor=center, above] {\scriptsize Upload} (Zenodo);
}


\end{tikzpicture}
\end{frame}

%--------------------------------------------------------------------------------
\begin{frame}[c]
\frametitle{Goals}

\begin{description}
\item[Reproducibility:] \textbf{IF} same experiment \\
identical parameters\\
same dataset\\
same model\\
\textbf{THEN} same results
\item[Traceability:] all the elements used in the analysis/experiment: \\
accessible\\
properly described\\
uniquely identified
\end{description}

\end{frame}

%-----------------------------------------------------------------------------------------------
\section{Zenodo platform}
%-----------------------------------------------------------------------------------------------

\begin{frame}

\begin{tikzpicture}[remember picture,overlay]
\fill[bluegher]
  (current page.west) rectangle (current page.south east);
\node[anchor=east] 
  at ([yshift=35pt, xshift=-20pt]current page.east) (title1)
  {\parbox[t]{.6\paperwidth}{\raggedleft%
   \usebeamerfont{sectiontitle2}{A closer look}}};
\node[anchor=east]
  at ([yshift=-35pt,xshift=-20pt]current page.east) (title2)
  {\parbox[t]{\textwidth}{\raggedleft%
   \usebeamerfont{sectiontitle1}\usebeamercolor[fg]{title}{to Zenodo}}};
\end{tikzpicture}

\end{frame}

%-----------------------------------------------------------------------------------------------

\begin{frame}[c]
\frametitle{Login: 3 options}

\begin{enumerate}
\item Use \faGithub~account
\item Use \aiOrcid~account
\item Create new \faUser~(Sign~Up)
\end{enumerate}

\begin{figure}
\centering
\includegraphics[width=.8\textwidth]{zenodo1}
\end{figure}

\end{frame}

%-----------------------------------------------------------------------------------------------

\begin{frame}[c]
\frametitle{Main page: linked accounts}

\begin{figure}
\centering
\includegraphics[width=.95\textwidth]{zenodo2}
\end{figure}

\end{frame}

%-----------------------------------------------------------------------------------------------

\begin{frame}[c]
\frametitle{Upload: you can drop anything}

\begin{figure}
\centering
\includegraphics[width=.9\textwidth]{zenodo4}
\end{figure}

In particular: Software \comment{ (ok for stable code)}
\end{frame}

%-----------------------------------------------------------------------------------------------
\begin{frame}[c]
\frametitle{Upload: add metadata}

\begin{figure}
\centering
\includegraphics[width=.9\textwidth]{zenodo_edit}
\end{figure}

\end{frame}

%-----------------------------------------------------------------------------------------------

\begin{frame}[c]
\frametitle{Upload metadata: licence}

\begin{figure}
\centering
\includegraphics[width=.9\textwidth]{zenodo_license}
\end{figure}
\vspace{.5cm}

Not necessarily Open!
\end{frame}

%-----------------------------------------------------------------------------------------------
\begin{frame}[c]
\frametitle{Upload metadata: funding!}

\begin{figure}
\centering
\includegraphics<1>[width=.9\textwidth]{zenodo_funding_zoom.png}
\includegraphics<2>[width=.9\textwidth]{openaire}
\end{figure}

\vfill

Common strategy for SeaDataCloud, ODIP, \ldots tools?
\end{frame}

%-----------------------------------------------------------------------------------------------

\begin{frame}[c]
\frametitle{Upload: finish}

\begin{figure}
\centering
\includegraphics[width=.9\textwidth]{zenodo_finish}
\end{figure}

\end{frame}

%-----------------------------------------------------------------------------------------------
\section{Zenodo and GitHub platform}
%-----------------------------------------------------------------------------------------------

\begin{frame}

\begin{tikzpicture}[remember picture,overlay]
\fill[bluegher]
  (current page.west) rectangle (current page.south east);
\node[anchor=east] 
  at ([yshift=35pt, xshift=-20pt]current page.east) (title1)
  {\parbox[t]{.6\paperwidth}{\raggedleft%
   \usebeamerfont{sectiontitle2}{Zenodo \& Github}}};
\node[anchor=east]
  at ([yshift=-35pt,xshift=-20pt]current page.east) (title2)
  {\parbox[t]{\textwidth}{\raggedleft%
   \usebeamerfont{sectiontitle1}\usebeamercolor[fg]{title}{working hand in hand}}};
\end{tikzpicture}

\end{frame}

%-----------------------------------------------------------------------------------------------

\begin{frame}[c]
\frametitle{Requirements}

\begin{enumerate}
\item \faGithub~or \aiOrcid account
\item Source code
\end{enumerate}

\vspace{1cm}

\begin{tikzpicture}
\node[fill=DarkOrange,text=black] (GitHub) at (4.5,.5) {\faGithub};
\node[fill=LightSkyBlue,text=black] (SourceCode) at (8,2) {Source Code};
\draw[->,>=stealth](SourceCode) edge[in=0,out=-90] node[anchor=center, above, sloped] {\scriptsize Upload} (GitHub);
\node[fill=DarkOrange,text=black] (ORCID) at (1,.5) {\aiOrcid};
\node[fill=LightSkyBlue,text=black] (Zenodo) at (4.5,2) {\includegraphics[height=.4cm]{figures/zenodo-black-200.png}}; 
\draw node[vertex] (Joint) at (2.5,0.5) {};
\draw[->,>=stealth](Software)(Joint) edge node[anchor=center, above,sloped] {\scriptsize Login} (Zenodo);
\draw[-] (ORCID) to (Joint);
\draw[-] (GitHub) to (Joint);
\node[fill=LightSkyBlue,text=black] (DOIcode) at (4.5,3.5) {\aiDoi};
\path[->,>=stealth]
(Zenodo) edge (DOIcode)
(SourceCode) edge node[anchor=center, above] {\scriptsize Upload} (Zenodo);
\end{tikzpicture}
\end{frame}


%-----------------------------------------------------------------------------------------------

\begin{frame}
\frametitle{Generating \aiDoi~for software releases}

\onslide*<1>{In Zenodo: click on the GitHub tab}
\onslide*<2>{In Zenodo: turn on the synchronisation\\ for the selected \faGithub~repositories}
\onslide*<3>{Go on your \faGithub~home page}
\onslide*<4>{In settings: allow third-party access}
\onslide*<5>{Open the selected project repository}
\onslide*<6>{Click on the \textit{Release} button}
\onslide*<7>{Fill in the information and \ldots}
\onslide*<8>{\ldots make the release}
\onslide*<9>{Check the project release on Zenodo and \ldots}
\onslide*<10>{\ldots get the \aiDoi~badge \comment{and celebrate}}

\vfill

\onslide*<1>{\includegraphics[width=.85\textwidth]{zenodo_github_tab}}
\onslide*<2>{\includegraphics[width=.85\textwidth]{zenodo3}}
\onslide*<3>{\includegraphics[width=.85\textwidth]{github1}}
\onslide*<4>{\includegraphics[width=.85\textwidth]{github3}}
\onslide*<5>{\includegraphics[width=.85\textwidth]{github_release1}}
\onslide*<6>{\includegraphics[width=.85\textwidth]{github_release2}}
\onslide*<7>{\includegraphics[width=.85\textwidth]{github_release3}}
\onslide*<8>{\includegraphics[width=.85\textwidth]{github_release4}}
\onslide*<9>{\includegraphics[width=.85\textwidth]{zenodo_release}}
\onslide*<10>{\includegraphics[width=.85\textwidth]{zenodo_release2}}

\end{frame}


%------------------------------------------------------------------------------------------------
\section{Use case 1: Diva}
%-----------------------------------------------------------------------------------------------

\begin{frame}

\begin{tikzpicture}[remember picture,overlay]
\fill[bluegher]
  (current page.west) rectangle (current page.south east);
\node[anchor=east] 
  at ([yshift=35pt, xshift=-20pt]current page.east) (title1)
  {\parbox[t]{.6\paperwidth}{\raggedleft%
   \usebeamerfont{sectiontitle2}{Use case I}}};
\node[anchor=east]
  at ([yshift=-35pt,xshift=-20pt]current page.east) (title2)
  {\parbox[t]{\textwidth}{\raggedleft%
   \usebeamerfont{sectiontitle1}\usebeamercolor[fg]{title}{get DOI for Diva releases}}};
\end{tikzpicture}

\end{frame}

%-----------------------------------------------------------------------------------------------

\begin{frame}[c]
\frametitle{Diva (simplified) development history}

\begin{description}
\item[1990’s:] Variational Interpolation Method (Fortran 77)\\
only 2D interpolations
\item[2006] SeaDataNet, code refactory and set of bash scripts 
\item[2007] \faPlug~with ODV \includegraphics[height=1.5ex]{odv_logo_small}
\item[2008] code in Subversion \includegraphics[height=2ex]{logo_svn}, distribution through GHER web page
\item[2009] new modules in Fortran 90\\
 for loops over depth and time 
\item[2012] new error calculation technique
\item[2017] switch from \includegraphics[height=2ex]{logo_svn} to \faGit, distribution via \href{https://github.com/gher-ulg/DIVA}{\faGithub}

\end{description}
\end{frame}

%-----------------------------------------------------------------------------------------------

\begin{frame}
\frametitle{From SVN to Git and GitHub}

\onslide*<1>{\textbf{Easy way}

\begin{enumerate}
\item Create a new repository with the latest release of the code
\item Lose all the history of the changes, the previous releases and developing branches\\
 \includegraphics[height=.5cm]{troll.png}
\end{enumerate}
}

\onslide*<2>{\textbf{Hard/conservative way}

\begin{enumerate}
\item Git repository whose structure reflects that of SVN\\
(trunk, branches, tags)
\item Use the \href{https://git-scm.com/book/en/v2/Git-and-Other-Systems-Migrating-to-Git	 }{\texttt{git-svn}} bridge\\
\item End up with a new GitHub repos with all the history
\end{enumerate}
}

\end{frame}

%-----------------------------------------------------------------------------------------------

\begin{frame}[t]
\frametitle{Use-case: Diva releases}

\begin{enumerate}
\item<1-> Switch from SVN to GitHub (conserving the history) 
\item<2-> Enable Diva repository on Zenodo
\item<3-> Edit the different \textit{tags} on GitHub to get DOI 
\item<4-> Now we can have\\
{\footnotesize
"\textit{The version used for the DIVA software is the 4.6.9,\\
doi: \href{http://doi.org/10.5281/zenodo.400968}{10.5281/zenodo.400968}}
}
\item<5-> Bonus: "\textit{cite as}" and social media
\end{enumerate}

\vspace{.25cm}

\onslide*<1>{\footnotesize
Resources:
\begin{itemize}
\item \url{https://git-scm.com/book/en/v2/Git-and-Other-Systems-Migrating-to-Git}
\item \url{http://john.albin.net/git/convert-subversion-to-git}
\item \url{https://www.atlassian.com/git/tutorials/migrating-overview}
\end{itemize}
}

\onslide*<2>{
\includegraphics[width=.9\textwidth]{zenodoDiva}
}

\onslide*<3>{
\includegraphics[width=.9\textwidth]{zenodoDiva2}
}

\onslide*<5>{
\includegraphics[width=.35\textwidth]{zenodoHowToCite}
}

\end{frame}


%------------------------------------------------------------------------------------------------
\section{Use case 2: SOCIB Glider toolbox}
%-----------------------------------------------------------------------------------------------

\begin{frame}

\begin{tikzpicture}[remember picture,overlay]
\fill[bluegher]
  (current page.west) rectangle (current page.south east);
\node[anchor=east] 
  at ([yshift=35pt, xshift=-20pt]current page.east) (title1)
  {\parbox[t]{.6\paperwidth}{\raggedleft%
   \usebeamerfont{sectiontitle2}{Use case II}}};
\node[anchor=east]
  at ([yshift=-35pt,xshift=-20pt]current page.east) (title2)
  {\parbox[t]{\textwidth}{\raggedleft%
   \usebeamerfont{sectiontitle1}\usebeamercolor[fg]{title}{SOCIB Glider toolbox}}};
\end{tikzpicture}

\end{frame}

%-----------------------------------------------------------------------------------------------

\begin{frame}[c]
\frametitle{Conclusions on Zendo}

\begin{enumerate}
\item Seamless integration with GitHub, login via ORCID
\item A big step toward reproducibility
\item A mature and user friend tool 
\end{enumerate}

\end{frame}

\end{document}