\documentclass[final,xcolor=table]{beamer}
\usepackage[english]{babel}
%\usepackage[latin1]{inputenc}
\usepackage[T1]{fontenc}
\usepackage{graphicx}
\usepackage{hyperref}
\usepackage{url}
\usepackage{tikz}
\usepackage{doi}
\usepackage{booktabs}
\usepackage{fontawesome}
\usepackage{academicons}
\usepackage{fontspec}


\setmainfont{Gillius ADF No2}

\newcommand{\logoheight}{1cm}

\DeclareGraphicsExtensions{.pdf,.png,.PNG,.JPG,.jpg,.jpeg,.gif}
\graphicspath{
{./figures/},
{./figures/scientificprofiles/}
{../../../figures4presentations/logo/},
}


\setbeamertemplate{navigation symbols}{}
\setbeamertemplate{items}[square] 
\setbeamertemplate{caption}[numbered]
%\setbeamerfont{caption}{size=\scriptsize,family=\it}

%\usefonttheme{professionalfonts} % using non standard fonts for beamer
%\usefonttheme{serif} % default family is serif

\usetikzlibrary{arrows,shapes,backgrounds}
\tikzstyle{every picture}+=[remember picture]
\tikzstyle{na} = [baseline=-.5ex]

\definecolor{bluegher}{HTML}{4E519F}
\definecolor{mygrey}{rgb}{0.75,.75,.75}
\definecolor{arrowcolor}{rgb}{0.1,0.1,0.1}
\definecolor{alertbg}{HTML}{FEFFBA}

\setbeamercolor{title}{fg=white}
\setbeamercolor{frametitle}{fg=white,bg=bluegher}
\setbeamercolor{institute}{fg=bluegher}
\setbeamercolor{structure}{fg=bluegher}
\setbeamercolor{item projected}{fg=black}

%\setbeamersize{text margin left=1cm}

\newlength{\logoH}
\setlength{\logoH}{1cm}

%--------------------------------
\hypersetup{bookmarksopen=true,
bookmarksnumbered=true,  
pdffitwindow=true, 
pdfstartview=Fit,
%pdfpagemode=FullScreen,
pdffitwindow=true,
pdftoolbar=true,
pdfmenubar=true,
pdfwindowui=true,
pdfauthor={C. Troupin, C. Muñoz, S. Watelet, A. Barth, J.M. Beckers},
bookmarksopenlevel=0,
colorlinks=true,
linkcolor=bluegher,anchorcolor=black,%
citecolor=bluegher,filecolor=black,%
menucolor=black,urlcolor=bluegher,%
pdfpageduration=1,%
}


\title{Notebooks for documenting work-flows}
%\subtitle{}
\author[C.~Troupin]{C.~Troupin$^{\star}$, A.~Barth$^{\star}$ \\ C.~Muñoz$^{\ast}$, S.~Watelet$^{\star}$,  \& J.-M.~Beckers$^{\star}$}

\institute{$^{\star}$GHER-University of Liège

$^{\ast}$Balearic Islands Coastal Ocean\\ Observing and Forecasting System}

\date{Galway (Ireland), 2-6 October, 2017}
\titlegraphic{\includegraphics[height=\logoheight]{logo_odip2}
\includegraphics[height=\logoheight]{logo_uliege.jpeg}\\
\includegraphics[height=\logoheight]{logo_gher}
\includegraphics[height=\logoheight]{logo_socib}}

\begin{document}

\begin{frame}
\maketitle
\end{frame}

\begin{frame}
\frametitle{Motivation}

Reproducibility

\end{frame}

%-------------------------------------------------------------------------------------------------------------------

\section{Notebooks}

\begin{frame}[t]
\frametitle{Notebooks: interactive computational environments}

\textit{Notebooks} combine:
\begin{enumerate}
\item code fragments that can be executed, 
\item text for the description of the application and 
\item figures illustrating the data or the results.
\end{enumerate}

\onslide*<1>{
\begin{figure}
\centering
\includegraphics[width=.8\textwidth]{notebook_ex}
\end{figure}
}

\onslide*<2>{
\vspace{1cm}
\parskip .5cm
\centering 
\it
"Digital Playground"

"Data Story Telling"

"Computational Narratives"
}

\onslide*<3>{
\vspace{1cm}
"Interactive notebooks: Sharing the code", Nature (2014)\\
\url{http://www.nature.com/news/interactive-notebooks-sharing-the-code-1.16261}
}

\end{frame}


%-------------------------------------------------------------------------------------------------------------------
\section{State of the art}
\begin{frame}

\begin{tikzpicture}[remember picture,overlay]
\fill[bluegher]
  (current page.west) rectangle (current page.south east);
\node[anchor=east] 
  at ([yshift=35pt, xshift=-20pt]current page.east) (title1)
  {\parbox[t]{.8\paperwidth}{\raggedleft%
   \usebeamerfont{sectiontitle2}{Interactive environments:}}};
\node[anchor=east]
  at ([yshift=-35pt,xshift=-20pt]current page.east) (title2)
  {\parbox[t]{\textwidth}{\raggedleft%
   \usebeamerfont{sectiontitle1}\usebeamercolor[fg]{title}{what exists today?}}};
\end{tikzpicture}

\end{frame}

%-------------------------------------------------------------------------------------------------------------------

\begin{frame}[t]
\frametitle{R-Markdown}
\parskip .5cm

\url{​http://rmarkdown.rstudio.com/​}

\onslide*<1>{
\includegraphics[width=\textwidth]{Rmarkdown}
}

\onslide*<2>{
Creation of dynamic, self-contained documents\\
with embedded chunks of code. 

Features of interest:
\begin{itemize}
\item Possible to export in journal (\url{​https://github.com/rstudio/rticles​}) or 
presentation formats
\item \LaTeX templates to ensure journal standards
\end{itemize}
}

\end{frame}

%-------------------------------------------------------------------------------------------------------------------

\begin{frame}[t]
\frametitle{Apache Zeppelin}

\parskip .25cm

\url{​https://zeppelin.apache.org/​}

\onslide*<1>{
\includegraphics[width=\textwidth]{Zeppelin}
}

\onslide*<2>{
Web-based notebook for\\ data-driven, interactive and collaborative documents.\\
Intended for \textit{big data} and large scale projects.

Features of interest:\begin{itemize}
\item Languages can be mixed in the same notebook
\item Users can write their own interpreter (\textit{language backend})
\end{itemize}
}

\end{frame}

%-------------------------------------------------------------------------------------------------------------------

\begin{frame}[t]
\frametitle{Jupyter}

\parskip .25cm

\url{​http://jupyter.org/}\hfill​ (stands for Julia - Python - R)

\onslide*<1>{
\includegraphics[width=\textwidth]{jupyter}
}

\onslide*<2>{
Web application for the creation and sharing of notebook-type documents.\\
Evolved from ​IPython​, a command shell for interactive computing (2001). 

Features of interest:
\begin{itemize}
\item More than 40 language \textit{kernels} available
\item Can be used as a multi-user server (\texttt{​jupyterhub​})\\
$\rightarrow$ avoid installation steps on several users' machine
\end{itemize} 
}

\end{frame}

%-------------------------------------------------------------------------------------------------------------------

\begin{frame}[t]
\frametitle{Beaker}

\parskip .25cm

\url{​http://beakernotebook.com/​}

\onslide*<1>{
\includegraphics[width=\textwidth]{beaker}
}

\onslide*<2>{
Notebook-style development environment for working interactively with large and complex datasets. 

Features of interest:
\begin{itemize}
\item Usage of different languages in different cells,\\ 
within the same notebook
\item Language manager
\end{itemize}
}

\end{frame}

%-------------------------------------------------------------------------------------------------------------------

\begin{frame}[t]
\frametitle{CoCalc}

\parskip .25cm

\url{​https://cocalc.com/​}\\ "\textit{Collaborative Calculation in the Cloud}" 

\onslide*<1>{
\includegraphics[width=\textwidth]{cocalc}
}

\onslide*<2>{
Web-based cloud computing platform, formerly called formerly called \texttt{SageMathCloud}. 

Features of interest:
\begin{itemize}
\item Support of many languages
\item Users to upload their file on the platform\\
 to be later read or processed
\end{itemize}
}
\end{frame}
%-------------------------------------------------------------------------------------------------------------------

\begin{frame}[c]
\frametitle{Comparison}

\Wider{
\begin{table}
\scriptsize
\begin{tabular}{C{1.5cm}L{1.5cm}L{1.5cm}L{1.5cm}L{1.5cm}L{1.5cm}}
\textbf{Tool name}		& R-Markdown	& jupyter	& beaker	& Cocalc & Zeppelin\\
\hline
\onslide<2->{\textbf{GitHub}			& \href{https://github.com/rstudio/rmarkdown}{rmarkdown} & 
\href{https://github.com/jupyter/notebook}{notebook} &
\href{https://github.com/twosigma/beakerx}{beakerx} &
\href{https://github.com/sagemathinc/cocalc}{cocalc} &
\href{https://github.com/apache/zeppelin}{zeppelin} \\
\hline}
\onslide<3->{\textbf{Languages}	& R, Python, SQL, Bash, Rcpp, Stan, JavaScript &
Julia, Python, R, Scala, Bash, Octave, Rubi, Fortran, PHP, \ldots&
Julia, Python, R, Javascript, C++, Torch, Scala, Bash, Octave, Rubi, Fortran, \ldots &
R​, Python, Octave​, Cython, ​Julia, Java, C/C++, Perl, Ruby &
Scala, Python, SparkSQL, Hive, Markdown \\
\hline}
\onslide<4->{
\textbf{Export formats} & 
HTML, PDF, MS Word, Beamer, HTML5 slides, \ldots & 
PDF, LaTeX, HMTL, Markdown, reST & 
Beaker format & 
& JSON \\
\hline}
\onslide<5->{
\textbf{Cloud deployment}	& -- & \cellcolor{Yellow}JupyterHub & Beaker Lab {\tiny (discontinued)} & -- & Yes \\
}
\end{tabular}
\end{table}
}

\end{frame}	


\begin{frame}[c]
\frametitle{Summary}

\begin{enumerate}

\item Most of the environments provides supports for many languages

\item Beaker is the only option allowing the mix of different languages\\
but its installation/utilisation are not trivial

\item ​JupyterHub is an option for the deployment on a server so that
multiple users can work at the same time using the same infrastructure
\end{enumerate}

\end{frame}

%--------------------------------------------------------------


\section{Notebooks to document workflows}
\begin{frame}

\begin{tikzpicture}[remember picture,overlay]
\fill[bluegher]
  (current page.west) rectangle (current page.south east);
\node[anchor=east] 
  at ([yshift=35pt, xshift=-20pt]current page.east) (title1)
  {\parbox[t]{.8\paperwidth}{\raggedleft%
   \usebeamerfont{sectiontitle2}{Notebook example:}}};
\node[anchor=east]
  at ([yshift=-35pt,xshift=-20pt]current page.east) (title2)
  {\parbox[t]{\textwidth}{\raggedleft%
   \usebeamerfont{sectiontitle1}\usebeamercolor[fg]{title}{divaND interpolation}}};
\end{tikzpicture}

\end{frame}

\begin{frame}[c]
\frametitle{A quick example of how to document workflow}

\href{./divand_example.slides.html}{Click here}
\end{frame}

\section{Conclusions}

\begin{frame}

\frametitle{Conclusions}

\begin{enumerate}
\item<1-> Notebooks are interactive computational environments\\
combining code, text, results, figures\ldots
\item<2-> Notebooks are not Virtual Research Environment,\\
but can be one of their components
\item<3-> Notebooks are not new (15 years)\\
but their use has evolved
\item<4-> Such a tool is great to document a workflow\\
Example: climatology production
\end{enumerate}

\end{frame}

\begin{frame}

\frametitle{Future work}

\begin{enumerate}
\item<1-> Examples using SeaDataCloud data
\item<2-> Application with data API \comment{(SOCIB, NOAA OneStop)}
\item<3-> Notebook citation \comment{See on Wednesday}
\end{enumerate}

\end{frame}


\end{document}