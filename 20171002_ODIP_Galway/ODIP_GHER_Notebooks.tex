\documentclass[final,xcolor=table]{beamer}
\usepackage[english]{babel}
%\usepackage[latin1]{inputenc}
\usepackage[T1]{fontenc}
\usepackage{graphicx}
\usepackage{hyperref}
\usepackage{url}
\usepackage{tikz}
\usepackage{doi}
\usepackage{booktabs}
\usepackage{fontawesome}
\usepackage{academicons}
\usepackage{fontspec}


\setmainfont{Gillius ADF No2}

\newcommand{\logoheight}{1cm}

\DeclareGraphicsExtensions{.pdf,.png,.PNG,.JPG,.jpg,.jpeg,.gif}
\graphicspath{
{./figures/},
{./figures/scientificprofiles/}
{../../../figures4presentations/logo/},
}


\setbeamertemplate{navigation symbols}{}
\setbeamertemplate{items}[square] 
\setbeamertemplate{caption}[numbered]
%\setbeamerfont{caption}{size=\scriptsize,family=\it}

%\usefonttheme{professionalfonts} % using non standard fonts for beamer
%\usefonttheme{serif} % default family is serif

\usetikzlibrary{arrows,shapes,backgrounds}
\tikzstyle{every picture}+=[remember picture]
\tikzstyle{na} = [baseline=-.5ex]

\definecolor{bluegher}{HTML}{4E519F}
\definecolor{mygrey}{rgb}{0.75,.75,.75}
\definecolor{arrowcolor}{rgb}{0.1,0.1,0.1}
\definecolor{alertbg}{HTML}{FEFFBA}

\setbeamercolor{title}{fg=white}
\setbeamercolor{frametitle}{fg=white,bg=bluegher}
\setbeamercolor{institute}{fg=bluegher}
\setbeamercolor{structure}{fg=bluegher}
\setbeamercolor{item projected}{fg=black}

%\setbeamersize{text margin left=1cm}

\newlength{\logoH}
\setlength{\logoH}{1cm}

%--------------------------------
\hypersetup{bookmarksopen=true,
bookmarksnumbered=true,  
pdffitwindow=true, 
pdfstartview=Fit,
%pdfpagemode=FullScreen,
pdffitwindow=true,
pdftoolbar=true,
pdfmenubar=true,
pdfwindowui=true,
pdfauthor={C. Troupin, C. Muñoz, S. Watelet, A. Barth, J.M. Beckers},
bookmarksopenlevel=0,
colorlinks=true,
linkcolor=bluegher,anchorcolor=black,%
citecolor=bluegher,filecolor=black,%
menucolor=black,urlcolor=bluegher,%
pdfpageduration=1,%
}


\title{Notebooks for documenting work-flows}
%\subtitle{}
\author[C.~Troupin]{C.~Troupin$^{\star}$, C.~Muñoz$^{\ast}$,\\S.~Watelet$^{\star}$, A.~Barth$^{\star}$ \& J.-M.~Beckers$^{\star}$}

\institute{$^{\star}$GHER-University of Liège

$^{\ast}$Balearic Islands Coastal Ocean\\ Observing and Forecasting System}

\date{Galway (Ireland), 2-6 October, 2017}
\titlegraphic{\includegraphics[height=\logoheight]{logo_odip2}
\includegraphics[height=\logoheight]{logo_uliege.jpeg}\\
\includegraphics[height=\logoheight]{logo_gher}
\includegraphics[height=\logoheight]{logo_socib}}

\begin{document}

\begin{frame}
\maketitle
\end{frame}

\begin{frame}
\frametitle{Motivation}

Reproducibility

\end{frame}

%-------------------------------------------------------------------------------------------------------------------

\section{Notebooks}

\begin{frame}[t]
\frametitle{Notebooks: interactive computational environments}

\textit{Notebooks} combine:
\begin{enumerate}
\item code fragments that can be executed, 
\item text for the description of the application and 
\item figures illustrating the data or the results.
\end{enumerate}

\onslide*<1>{
\begin{figure}
\centering
\includegraphics[width=.8\textwidth]{notebook_ex}
\end{figure}
}

\onslide*<2>{
\vspace{1cm}
\parskip .5cm
\centering 
\it
"Digital Playground"

"Data Story Telling"

"Computational Narratives"
}

\onslide*<3>{
\vspace{1cm}
"Interactive notebooks: Sharing the code", Nature (2014)\\
\url{http://www.nature.com/news/interactive-notebooks-sharing-the-code-1.16261}
}

\end{frame}


%-------------------------------------------------------------------------------------------------------------------


\begin{frame}[t]
\frametitle{Interactive environments: what exists today?}


\end{frame}




\end{document}